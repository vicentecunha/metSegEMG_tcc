Caro professor Balbinot,

Percebes que não fiz avanços significativos no texto, pois preocupei-me em trabalhar nas implementações dos métodos. Seguem o próximos passos que pretendo seguir para finalizar este trabalho.
\begin{itemize}
	\item Falar com Vinicius para obter o cálculo de frequência mediana que ele está utilizando.
	\item Falar com Fernanda sobre o processo de aquisição da database do laboratório.
	\item Obtenção de parâmetros para os métodos.
	\item Implementar a rede neural e analisar resultados.
\end{itemize}

Sobre o preprocessamento: primeiramente, retifico os canais do sinal (utilizando \emph{abs()}). Para lidar com a existência de múltiplos canais do sinal de EMG, realizo a soma de todos os canais, obtendo assim um único sinal \emph{$x_{sum}$}. Posteriormente, filtro este sinal \emph{$x_{sum}$} com um FIR com frequência de corte 20 Hz. Obtenho as posições de segmentos do sinal original (anterior à soma) aplicando os métodos (MTD1-4) a este sinal \emph{$x_{sum}$} filtrado. Estás de acordo com os passos realizado?

Estou receoso sobre como justificar o uso de um FIR nesta baixa frequência, visto que sinais de EMG têm banda de até 500 Hz. Porém, a forma do sinal obtido pela soma dos canais é relacionada aos movimentos realizados pela pessoa, que têm esta frequência menor associada. Além disto, para os métodos de segmentação, a ocorrência de \emph{spikes} no sinal afeta negativamente os resultados dos métodos, e a filtragem em 20 Hz atenua tais ocorrências. Consideras, portanto, devidamente justificável o uso de FIR em 20 Hz?

Ainda estou decidindo a melhor estratégia para determinar os parâmetros dos métodos (de formas que não sejam completamente empíricas). Tenho em mente implementar os métodos realizando \emph{``sweeping''} dos parâmetros, utilizando número de segmentos obtidos e distâncias entre eles como métrica da determinação dos melhores parâmetros. Tens alguma sugestão a dar a respeito desta ou outra estratégia?

Sobre trabalhos futuros, gostaria de sugerir a adaptação dos métodos, que são de natureza essencialmente não-causal (\emph{offline}), para métodos causais (\emph{online}). De acordo?

Grato,
Vicente