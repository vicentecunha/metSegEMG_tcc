% Set up a few colours
\colorlet{lcfree}{green}
\colorlet{lcnorm}{blue}
\colorlet{lccong}{red}
% -------------------------------------------------
% Set up a new layer for the debugging marks, and make sure it is on
% top
\pgfdeclarelayer{marx}
\pgfsetlayers{main,marx}
% A macro for marking coordinates (specific to the coordinate naming
% scheme used here). Swap the following 2 definitions to deactivate
% marks.
\providecommand{\cmark}[2][]{%
  \begin{pgfonlayer}{marx}
    \node [nmark] at (c#2#1) {#2};
  \end{pgfonlayer}{marx}
  } 
\providecommand{\cmark}[2][]{\relax} 
% -------------------------------------------------
% Start the picture
\begin{figure}[htb]
	\caption{\label{flow:MTDs} Fluxograma geral para os métodos de segmentação MTD1 - MTD4.}
	\begin{center}
		\begin{tikzpicture}[%
			>=triangle 60,              % Nice arrows; your taste may be different
			start chain=going below,    % General flow is top-to-bottom
			node distance=6mm and 60mm, % Global setup of box spacing
			every join/.style={norm},   % Default linetype for connecting boxes
			]
		% ------------------------------------------------- 
		% A few box styles 
		% <on chain> *and* <on grid> reduce the need for manual relative
		% positioning of nodes
		\tikzset{
		  base/.style={draw, on chain, on grid, align=center, minimum height=4ex},
		  proc/.style={base, rectangle, text width=8em},
		  test/.style={base, diamond, aspect=2, text width=8em},
		  term/.style={proc, rounded corners},
		  % coord node style is used for placing corners of connecting lines
		  coord/.style={coordinate, on chain, on grid, node distance=6mm and 25mm},
		  % nmark node style is used for coordinate debugging marks
		  nmark/.style={draw, cyan, circle, font={\sffamily\bfseries}},
		  % -------------------------------------------------
		  % Connector line styles for different parts of the diagram
		  norm/.style={->, draw, lcnorm},
		  free/.style={->, draw, lcfree},
		  cong/.style={->, draw, lccong},
		  it/.style={font={\small\itshape}}
		}
		% -------------------------------------------------
		% Node placement: column 1
		\node [proc, fill=lcfree!25] (pre) {Preprocessamento (retificação e normalização)};
		\node [proc, fill=lcfree!25, right=of pre, xshift=-4em, join] (id) {Identificação de segmentos para cada canal utilizando MTD\#};
		\node [proc, fill=lcfree!25, right=of id, xshift=-4em, join] (dbscan) {Grupamento de posições de segmentos utilizando DBSCAN};
		\node [proc, fill=lcfree!25, right=of dbscan, xshift=-4em, join] {Segmentação do sinal nas médias das posições agrupadas};
		% -------------------------------------------------
		\end{tikzpicture}
	\end{center}
\end{figure}
