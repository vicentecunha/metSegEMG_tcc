%------------------------------------------------------------------------------		
% Styles:
%------------------------------------------------------------------------------
\tikzstyle{io} = [circle, minimum width=0.1cm, draw=black, fill=black]
\tikzstyle{sum} = [circle, minimum width=1cm, draw=black]
\tikzstyle{func}= [rectangle, minimum width=1cm, minimum height=1cm, draw=black]

\begin{figure}[htb]
	\caption{\label{fig:neuron} Diagrama representativo do modelo de um neurônio artificial genérico.}
	\begin{center}
		\begin{tikzpicture}[>=triangle 60]
%------------------------------------------------------------------------------
% Node placement
%------------------------------------------------------------------------------
			\node[io](x1){}; \node[left=of x1, xshift=2em]{$in_1$};
			\node[io, below=of x1](x2){}; \node[left=of x2, xshift=2em]{$in_2$};
			\node[io, below=of x2](x3){}; \node[left=of x3, xshift=2em]{$in_3$};
			\node[below=of x3, yshift=2em]{...};			
			\node[io, below=of x3](xn){}; \node[left=of xn, xshift=2em]{$in_n$};
			
			\node[sum, right=of x2, xshift = 2em, yshift = -1.5em](sum){$\sum$};
			
			\node[io, below=of sum](B){}; \node[right=of B, xshift=-2em]{B};
			
			\node[func, right=of sum](f){$f$}; \node[above=of f, align=left, yshift=-2em]{Função de \\ Ativação};
			
			\node[right=of f](out){}; \node[right=of out, xshift=-3em]{$out$};
			
%------------------------------------------------------------------------------
% Connections
%------------------------------------------------------------------------------
			\draw [->,black] (x1) -- (sum);
			\path (x1) to node [near start, xshift=1em]{$w_1$}(sum);
			
			\draw [->,black] (x2) -- (sum);                                
			\path (x2) to node [near start, xshift=1em, yshift=0.3em]{$w_2$}(sum);
			
			\draw [->,black] (x3) -- (sum);                                
			\path (x3) to node [near start, xshift=1em, yshift=-0.1em]{$w_3$}(sum);
			
			\draw [->,black] (xn) -- (sum);                                
			\path (xn) to node [near start, xshift=1em]{$w_n$}(sum);
			
			\draw [->,black] (B) -- (sum);
			\draw [->,black] (sum) -- (f);
			\draw [->,black] (f) -- (out);

		\end{tikzpicture}
	\end{center}
%	\legend{Fonte: adaptado de \cite{Tanikic2012}}
\end{figure}
