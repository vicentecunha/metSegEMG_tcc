% Set up a few colours
\colorlet{lcfree}{green}
\colorlet{lcnorm}{blue}
\colorlet{lccong}{red}
% -------------------------------------------------
% Set up a new layer for the debugging marks, and make sure it is on
% top
\pgfdeclarelayer{marx}
\pgfsetlayers{main,marx}
% A macro for marking coordinates (specific to the coordinate naming
% scheme used here). Swap the following 2 definitions to deactivate
% marks.
\providecommand{\cmark}[2][]{%
  \begin{pgfonlayer}{marx}
    \node [nmark] at (c#2#1) {#2};
  \end{pgfonlayer}{marx}
  } 
\providecommand{\cmark}[2][]{\relax} 
% -------------------------------------------------
% Start the picture
\begin{figure}[htb]
	\caption{\label{fig_mtd4flux}Fluxograma representativo do MTD4.}
	\begin{center}
		\begin{tikzpicture}[%
			>=triangle 60,              % Nice arrows; your taste may be different
			start chain=going below,    % General flow is top-to-bottom
			node distance=6mm and 60mm, % Global setup of box spacing
			every join/.style={norm},   % Default linetype for connecting boxes
			]
		% ------------------------------------------------- 
		% A few box styles 
		% <on chain> *and* <on grid> reduce the need for manual relative
		% positioning of nodes
		\tikzset{
		  base/.style={draw, on chain, on grid, align=center, minimum height=4ex},
		  proc/.style={base, rectangle, text width=10em},
		  test/.style={base, diamond, aspect=2, text width=6em},
		  term/.style={proc, rounded corners},
		  % coord node style is used for placing corners of connecting lines
		  coord/.style={coordinate, on chain, on grid, node distance=6mm and 25mm},
		  % nmark node style is used for coordinate debugging marks
		  nmark/.style={draw, cyan, circle, font={\sffamily\bfseries}},
		  % -------------------------------------------------
		  % Connector line styles for different parts of the diagram
		  norm/.style={->, draw, lcnorm},
		  free/.style={->, draw, lcfree},
		  cong/.style={->, draw, lccong},
		  it/.style={font={\small\itshape}}
		}
		% -------------------------------------------------
		% Node placement: column 1
		\node [term, densely dotted, it, fill=lccong!25] (init) {INÍCIO};
		\node [proc, fill=lcnorm!25, join] {$w_0 = 0$};
		\node [proc, fill=lcfree!25, join] (W) {Janela de comprimento $W$, iniciando em $w_0$};
		\node [proc, fill=lcnorm!25, join] (w0) {$w_0 \gets w_0 + 1$};
		\node [test, join] (t1) {$w_0 + W > L$};
		% -------------------------------------------------
		% Node placement: column 2
		\node [test, right=of t1, fill=lcfree!25, xshift = 2em] (maxt1) {Máximo do sinal contido na janela $< T$};
		\node [proc, fill=lcfree!25] (BEP) {$w_0$ é BEP};
		\node [proc, fill=lcnorm!25, join] (w01) {$w_0 \gets w_0 + 1$};
		\node [test, join] (t2) {$w_0 + W > L$};
		\node [test, fill=lcfree!25] (maxt2) {Máximo do sinal contido na janela $< T$};
		\node [proc, fill=lcfree!25] (EEP) {$w_0+W$ é EEP};
		% -------------------------------------------------
		% Node placement: column 1
		\node [proc, fill=lcfree!25, left=of maxt2, xshift = -2em] (fin) {Segmentos formados pelos BEPs e EEPs identificados};
		\node [term, densely dotted, it, fill=lccong!25, join] (fim) {FIM};
		% -------------------------------------------------
		% Coordinates
		\node [coord, right=of maxt1, xshift = 6em] (c1) {};
		\node [coord, below=of t1, yshift = -13.5em] (c2) {};
		\node [coord, right=of maxt2, xshift = 4em] (c3) {};
%		% -------------------------------------------------
%		% Test connections
		\path (t1.east) to node [near start, yshift=1em] {$não$} (maxt1.west);
			\draw [o->,lcnorm] (t1.east) -- (maxt1.west);
		\path (t1.south) to node [near start, xshift=1em, yshift=4em] {$sim$} (fin);
			\draw [*->,lcnorm] (t1.south) -- (fin);
		\path (maxt1.south) to node [near start, xshift=1em] {$não$} (BEP);
			\draw [o->,lcnorm] (maxt1.south) -- (BEP);
		\path (maxt1.east) to node [near start, xshift = 2em] {$sim$} (BEP);
			\draw [*->,lcnorm] (maxt1.east) -- (c1);
		\path (t2.south) to node [near start, xshift=1em] {$não$} (maxt2);
			\draw [o->,lcnorm] (t2.south) -- (maxt2);
		\path (t2.west) to node [near start, yshift=1em] {$sim$} (c2);
			\draw [*->,lcnorm] (t2.west) -- (c2);
		\path (maxt2.east) to node [near start, yshift=1em] {$não$} (c3);
			\draw [o->,lcnorm] (maxt2.east) -- (c3) |- (w01);
		\path (maxt2.south) to node [near start, xshift=1em] {$sim$} (EEP);
			\draw [*->,lcnorm] (maxt2.south) -- (EEP);
			\draw [->,lcnorm] (EEP) -| (c1) |- (w0);
		
		% -------------------------------------------------
		\end{tikzpicture}
	\end{center}
\end{figure}
