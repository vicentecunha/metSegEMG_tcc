\begin{table}[htb]
\IBGEtab{%
	\caption{Parâmetros utilizados para definir o MTD3.}%
	\label{tab:mtd3params}
}{%
	\begin{tabular}{ccc}
	\toprule
	Nome & Descrição \\
	\midrule \midrule
	$x$ & Sinal a ser segmentado \\
	\midrule
	$l_{min}$ & Distância mínima entre BEPs e EEPs de um mesmo segmento \\
	\midrule
	$l_{máx}$ & Distância máxima entre BEPs e EEPs de um mesmo segmento\\
	\midrule
	$W$ & Comprimento da janela deslizante utilizada pelo método \\
	\midrule
	$w_0$ & Número da amostra mais a esquerda da janela. Determina a posição instantânea da janela \\
	\midrule
	$step$ & Número de amostras para incrementar $w_0$ antes de novo cálculo de variação total \\
	\midrule
	$V$ & Variação total do sinal $x$ contido na janela deslizante \\
	\midrule
	$\beta$ & Declividade média do sinal $x$ contido na janela deslizante \\
	\midrule
	$B$ & Valor limite para declividade média que determina um BEP\\
	\midrule
	$\gamma$ & Módulo da variação total do sinal $x$ contido na janela deslizante \\
	\midrule
	$C$ & Valor limite para variação total que determina um EEP\\
	\bottomrule
\end{tabular}%
}{%
}
\end{table}