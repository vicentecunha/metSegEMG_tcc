\section{Implementação de Métodos de Segmentação}

Esta seção descreve a implementação dos métodos de segmentação desenvolvidos. O fluxograma da Figura \ref{flow:MTDs} apresenta de forma resumida os passos comuns aos quatro métodos, que serão explanados nas subseções seguintes. Os códigos em Matlab criados para os quatro métodos são apresentados nos Apêndices A - D.

% Set up a few colours
\colorlet{lcfree}{green}
\colorlet{lcnorm}{blue}
\colorlet{lccong}{red}
% -------------------------------------------------
% Set up a new layer for the debugging marks, and make sure it is on
% top
\pgfdeclarelayer{marx}
\pgfsetlayers{main,marx}
% A macro for marking coordinates (specific to the coordinate naming
% scheme used here). Swap the following 2 definitions to deactivate
% marks.
\providecommand{\cmark}[2][]{%
  \begin{pgfonlayer}{marx}
    \node [nmark] at (c#2#1) {#2};
  \end{pgfonlayer}{marx}
  } 
\providecommand{\cmark}[2][]{\relax} 
% -------------------------------------------------
% Start the picture
\begin{figure}[htb]
	\caption{\label{flow:MTDs} Fluxograma geral para os métodos de segmentação MTD1 - MTD4.}
	\begin{center}
		\begin{tikzpicture}[%
			>=triangle 60,              % Nice arrows; your taste may be different
			start chain=going below,    % General flow is top-to-bottom
			node distance=6mm and 60mm, % Global setup of box spacing
			every join/.style={norm},   % Default linetype for connecting boxes
			]
		% ------------------------------------------------- 
		% A few box styles 
		% <on chain> *and* <on grid> reduce the need for manual relative
		% positioning of nodes
		\tikzset{
		  base/.style={draw, on chain, on grid, align=center, minimum height=4ex},
		  proc/.style={base, rectangle, text width=8em},
		  test/.style={base, diamond, aspect=2, text width=8em},
		  term/.style={proc, rounded corners},
		  % coord node style is used for placing corners of connecting lines
		  coord/.style={coordinate, on chain, on grid, node distance=6mm and 25mm},
		  % nmark node style is used for coordinate debugging marks
		  nmark/.style={draw, cyan, circle, font={\sffamily\bfseries}},
		  % -------------------------------------------------
		  % Connector line styles for different parts of the diagram
		  norm/.style={->, draw, lcnorm},
		  free/.style={->, draw, lcfree},
		  cong/.style={->, draw, lccong},
		  it/.style={font={\small\itshape}}
		}
		% -------------------------------------------------
		% Node placement: column 1
		\node [proc, fill=lcfree!25] (pre) {Preprocessamento (retificação e normalização)};
		\node [proc, fill=lcfree!25, right=of pre, xshift=-4em, join] (id) {Identificação de segmentos para cada canal utilizando MTD\#};
		\node [proc, fill=lcfree!25, right=of id, xshift=-4em, join] (dbscan) {Grupamento de posições de segmentos utilizando DBSCAN};
		\node [proc, fill=lcfree!25, right=of dbscan, xshift=-4em, join] {Segmentação do sinal nas médias das posições agrupadas};
		% -------------------------------------------------
		\end{tikzpicture}
	\end{center}
\end{figure}



\subsection {Preprocessamento}
\subsubsection{Retificação de Sinal}

Os sinais de eletromiografia para ambas as bases de dados (Ninapro e IEE) são armazenados mantendo sua polaridade original (\emph{i.e.} amostras do sinal podem assumir valores positivos e negativos). Primeiramente, realiza-se a retificação completa dos sinais tomando o módulo dos valores amostrados (função \emph{abs()}). A retificação completa do sinal mantém sua energia e é fundamental para a implementação dos métodos de segmentação aqui desenvolvidos. A Figura \ref{fig:rectification} exemplifica o resultado esperado para a retificação completa de um trecho de sinal de eletromiografia.

\begin{figure}[htb]
	\caption{\label{fig:rectification} Retificação completa de trecho de sinal de eletromiografia.}
	\begin{center}
	    \includegraphics[width=0.75\linewidth]{./img/placeholder.png}
	\end{center}
	%\legend{Fonte: adaptado de BASMAJIAN \& DE LUCA, 1985}
\end{figure}

\subsubsection{Normalização}

Os sinais para cada canal de aquisição são normalizados de acordo com seu valor máximo, de modo que seu novo valor máximo seja unitário, a partir da equação \ref{eq:x_normalization}, onde $x$ é o sinal original para um canal e $x_{norm}$ é sua versão normalizada. A normalização de canais faz com que os parâmetros utilizados pelos métodos de segmentação sejam relativos ao valor máximo do sinal, possibilitando a implementação para diferentes voluntários. A Figura \ref{fig:normalization} exemplifica a normalização para três canais de um trecho de sinal já retificado.

\begin{equation}
	\label{eq:x_normalization}
	x_{norm} = \frac{x}{max(x)}
\end{equation}

\begin{figure}[htb]
	\caption{\label{fig:normalization}Normalização de canais de eletromiografia de acordo com seu valor máximo.}
	\begin{center}
	    \includegraphics[width=0.75\linewidth]{./img/placeholder.png}
	\end{center}
	%\legend{Fonte: adaptado de BASMAJIAN \& DE LUCA, 1985}
\end{figure}


\subsection{Implementação dos métodos de segmentação}
\subsubsection{Parâmetros utilizados}

Cada método de segmentação MTD1 - MTD4 apresenta um conjunto de parâmetros ajustáveis. Após investigações iniciais das segmentações obtidas com diferentes valores de parâmetros, fixou-se alguns destes parâmetros e realizou-se uma lista de valores a serem explorados em cada método. Os métodos de segmentação foram então aplicados aos sinais das bases de dados considerando as diferentes possibilidades de combinação para tais valores. A Tabela \ref{tab:combinacoes} apresenta os parâmetros ajustáveis de cada método e sua lista de valores explorada.

\begin{table}[htb]
\IBGEtab{\caption{\label{tab:combinacoes}Parâmetros ajustáveis para os métodos de segmentação.}}
{
	\begin{tabular}{cllc}
		\toprule
		Método 					& Parâmetros	& Valores utilizados						& Número total de combinações \\
		\midrule \midrule					
		\multirow{4}{*}{MTD1}	& $l$			& $10 \times 10^3$							& \multirow{4}{*}{16} \\
								& $r_{target}$	& $5,6 \times 10^{-5}$						& \\
								& $q$			& $[0,8 \quad 0,85 \quad 0,9 \quad 0,95]$	& \\
								& $T_{lim`}$	& $[0,05 \quad 0,1 \quad 0,15 \quad 0,2]$	& \\
		\midrule
		\multirow{4}{*}{MTD2}	& $l$			& $10 \times 10^3$							& \multirow{4}{*}{27} \\
								& $A$			& $[20 \quad 30 \quad 40]$					& \\
								& $B$			& $[2 \quad 5 \quad 8]$						& \\
								& $C$			& $[2 \quad 5 \quad 8]$						& \\
		\midrule					
		\multirow{6}{*}{MTD3}	& $W$			& $5 \times 10^3$							& \multirow{6}{*}{16} \\
								& $step$		& $500$										& \\
								& $l_{mín}$		& $7,5 \times 10^3$							& \\
								& $l_{máx}$		& $12,5 \times 10^3$						& \\
								& $B$			& [0,1 \quad 0,2 \quad 0,3 \quad 0,4]		& \\
								& $C$			& [-0,1 \quad -0,2 \quad -0,3 \quad -0,4]	& \\
		\midrule					
		\multirow{2}{*}{MTD4}	& \textcolor{red}{TODO}	& \textcolor{red}{TODO}				& \multirow{2}{*}{\textcolor{red}{TODO}} \\
								& \textcolor{red}{TODO}	& \textcolor{red}{TODO}				& \\
		\bottomrule
	\end{tabular}
}{}
\end{table}


\subsubsection{Identificação de segmentos utilizando \emph{k-means}}

Os sinais para ambas as bases de dados são compostos por 12 canais de aquisição. Os métodos de segmentação são implementados individualmente aos doze canais. Para os métodos MTD1 e MTD2, as posições centrais dos segmentos obtidas em cada canal são registradas, enquanto que para os métodos MTD3 e MT4 registra-se as posições de BEPs e EEPs. Tais posições são agrupadas utilizando o método de \emph{clustering k-means}. Este agrupamento permite a identificação dos segmentos obtidos nos diferentes canais que referem-se a um mesmo trecho de aumento da atividade muscular. Tomando a média de cada grupo, o sinal original pode então ser segmentado, de forma que os segmentos mantém coerência temporal entre canais. A Figura \ref{fig:kmeans1} exemplifica o grupamento das posições centrais de segmentos nos métodos MTD1 e MTD2 e a Figura \ref{fig:kmeans2} exemplifica o grupamento de BEPs e EEPs nos métodos MTD3 e MTD4.

\begin{figure}[htb]
	\caption{\label{fig:kmeans1}\emph{Clustering} por \emph{k-means} dos centros de segmentos obtidos pelos métodos MTD1 e MTD2.}
	\begin{center}
	    \includegraphics[width=0.75\linewidth]{./img/placeholder.png}
	\end{center}
	%\legend{Fonte: adaptado de BASMAJIAN \& DE LUCA, 1985}
\end{figure}

\begin{figure}[htb]
	\caption{\label{fig:kmeans2}\emph{Clustering} por \emph{k-means} de BEPs e EEPs de segmentos obtidos pelos métodos MTD3 e MTD4.}
	\begin{center}
	    \includegraphics[width=0.75\linewidth]{./img/placeholder.png}
	\end{center}
	%\legend{Fonte: adaptado de BASMAJIAN \& DE LUCA, 1985}
\end{figure}

%==============================================================================
\section{Rede Neural Artificial}

Esta seção descreve a utilização de RNA para classificação dos segmentos obtidos de acordo com movimentos de interesse. O processo de classificação é representado pelo fluxograma da Figura \ref{flow:RNAresumo}, que será explanado nas subseções seguintes.

\begin{figure}[htb]
	\caption{\label{fig:kmeans2}\emph{Clustering} por \emph{k-means} de BEPs e EEPs de segmentos obtidos pelos métodos MTD3 e MTD4.}
	\begin{center}
	    \includegraphics[width=0.75\linewidth]{./img/placeholder.png}
	\end{center}
	%\legend{Fonte: adaptado de BASMAJIAN \& DE LUCA, 1985}
\end{figure}


\subsection{Características utilizadas como preditores}

Os preditores (as ``entradas'') utilizados pela RNA são o valor \emph{rms}, a variância e a frequência mediana dos segmentos de sinal obtidos pelos métodos MTD1 - MTD4.

O valor \emph{rms} ou \emph{root mean square} de um sinal (em Matlab, função \emph{rms()}) é dada pela equação \ref{eq:rms}. 

\subsection{Treinamento, Validação e Teste}
