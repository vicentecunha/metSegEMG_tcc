Este capítulo divide-se entre a implmentação dos métodos de segmentação MTD1 - MTD4 e a obtenção de resultados de classificação utilizando RNA. O fluxograma da Figura \ref{fig:panorama} apresenta uma visão geral da metodologia utilizada, que será explanada em sequência.

\begin{figure}[!htb]
	\label{fig:rectification}
	\caption{Fluxograma geral da metodologia experimental.}
	\begin{center}
	    \includegraphics[width=0.75\linewidth]{./img/placeholder.png}
	\end{center}
	%\legend{Fonte: adaptado de BASMAJIAN \& DE LUCA, 1985}
\end{figure}


% ---
\section{Implementação de Métodos de Segmentação}
% ---

Esta seção descreve a implementação dos métodos de segmentação desenvolvidos. 

\subsection {Preprocessamento}
\subsubsection{Retificação de Sinal}

Os sinais de eletromiografia para ambas as bases de dados (Ninapro e IEE) são armazenados mantendo sua polaridade original (\emph{i.e.} amostras do sinal podem assumir valores positivos e negativos). Primeiramente, realiza-se a retificação completa dos sinais tomando o módulo dos valores amostrados (função \emph{abs()}). A retificação completa do sinal mantém toda sua energia e é fundamental para a implementação dos métodos de segmentação aqui desenvolvidos. A Figura \ref{fig:rectification} exemplifica o resultado esperado para a retificação completa de um trecho de sinal de eletromiografia.

\begin{figure}[!htb]
	\label{fig:rectification}
	\caption{Retificação completa de trecho de sinal de eletromiografia.}
	\begin{center}
	    \includegraphics[width=0.75\linewidth]{./img/placeholder.png}
	\end{center}
	%\legend{Fonte: adaptado de BASMAJIAN \& DE LUCA, 1985}
\end{figure}

\subsection{Normalização}

Os sinais para cada canal de aquisição são normalizados de acordo com seu valor máximo, de modo que seu novo valor máximo seja unitário, a partir da equação \ref{eq:normalization}, onde $x$ é o sinal original para um canal e $x_{norm}$ é sua versão normalizada. A normalização de canais faz com que os parâmetros utilizados pelos métodos de segmentação sejam relativos ao valor máximo do sinal, possibilitando a implementação para diferentes voluntários. A Figura \ref{fig:normalization} exemplifica a retificação para três canais de um trecho de sinal já retificado.

\begin{equation}
	\label{eq:x_normalization}
	x_{norm} = \frac{x}{max(x)}
\end{equation}

\begin{figure}[!htb]
	\label{fig:normalization}
	\caption{Normalização de canais de eletromiografia de acordo com seu valor máximo.}
	\begin{center}
	    \includegraphics[width=0.75\linewidth]{./img/placeholder.png}
	\end{center}
	%\legend{Fonte: adaptado de BASMAJIAN \& DE LUCA, 1985}
\end{figure}

% ---
\section{Parâmetros utilizados nos métodos}
% ---

Cada método de segmentação MTD1 - MTD4 apresenta um conjunto de parâmetros ajustáveis. Após investigações iniciais das segmentações obtidas com diferentes valores de parâmetros, fixou-se alguns destes parâmetros e realizou-se uma lista de valores a serem explorados em cada método. Os métodos de segmentação foram então aplicados aos sinais das bases de dados considerando as diferentes possibilidades de combinação para tais valores. A Tabela \ref{tab:combinacoes} apresenta os parâmetros ajustáveis de cada método e sua lista de valores explorada.

\begin{table}[htb]
\IBGEtab{\caption{\label{tab:combinacoes}Parâmetros ajustáveis para os métodos de segmentação.}}
{
	\begin{tabular}{cllc}
		\toprule
		Método 					& Parâmetros	& Valores utilizados						& Número total de combinações \\
		\midrule \midrule					
		\multirow{4}{*}{MTD1}	& $l$			& $10 \times 10^3$							& \multirow{4}{*}{16} \\
								& $r_{target}$	& $5,6 \times 10^{-5}$						& \\
								& $q$			& $[0,8 \quad 0,85 \quad 0,9 \quad 0,95]$	& \\
								& $T_{lim`}$	& $[0,05 \quad 0,1 \quad 0,15 \quad 0,2]$	& \\
		\midrule
		\multirow{4}{*}{MTD2}	& $l$			& $10 \times 10^3$							& \multirow{4}{*}{27} \\
								& $A$			& $[20 \quad 30 \quad 40]$					& \\
								& $B$			& $[2 \quad 5 \quad 8]$						& \\
								& $C$			& $[2 \quad 5 \quad 8]$						& \\
		\midrule					
		\multirow{6}{*}{MTD3}	& $W$			& $5 \times 10^3$							& \multirow{6}{*}{16} \\
								& $step$		& $500$										& \\
								& $l_{mín}$		& $7,5 \times 10^3$							& \\
								& $l_{máx}$		& $12,5 \times 10^3$						& \\
								& $B$			& [0,1 \quad 0,2 \quad 0,3 \quad 0,4]		& \\
								& $C$			& [-0,1 \quad -0,2 \quad -0,3 \quad -0,4]	& \\
		\midrule					
		\multirow{2}{*}{MTD4}	& \textcolor{red}{TODO}	& \textcolor{red}{TODO}				& \multirow{2}{*}{\textcolor{red}{TODO}} \\
								& \textcolor{red}{TODO}	& \textcolor{red}{TODO}				& \\
		\bottomrule
	\end{tabular}
}{}
\end{table}


% ---
\section{Rede Neural Artificial}
% ---



%---
\section{Características utilizadas como preditores}
%---


\subsection{Treinamento, Validação e Teste}
